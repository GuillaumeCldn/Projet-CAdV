\documentclass[12pt,oneside]{scrreprt}

% ---------------------------------------------------------
% PACKAGES
% ---------------------------------------------------------
\usepackage[utf8]{inputenc}
\usepackage[T1]{fontenc}
\usepackage[french]{babel}
\usepackage{graphicx}
\usepackage{subcaption}
\usepackage{amsmath}
\usepackage{siunitx}
\usepackage{booktabs}
\usepackage[hidelinks]{hyperref}
\usepackage{lmodern}
\usepackage{microtype}
\usepackage{float}
\usepackage{fmtcount}
\usepackage{tikz}
\usetikzlibrary{arrows.meta, matrix, decorations.pathreplacing, calc}
\usepackage{wrapfig}

% ---------------------------------------------------------
% MISE EN PAGE
% ---------------------------------------------------------
\usepackage[margin=2.5cm]{geometry}

\setlength{\parskip}{0.85em}
\setlength{\parindent}{1.2em}
\linespread{1.15}

% ---------------------------------------------------------
% TITRES (méthode KOMA-Script)
% ---------------------------------------------------------
\RedeclareSectionCommand[
beforeskip=1.8em plus 0.2em minus 0.2em,
afterskip=1em,
font=\normalfont\huge\bfseries]{chapter}

\RedeclareSectionCommand[
beforeskip=1.4em plus 0.2em minus 0.2em,
afterskip=0.8em,
font=\normalfont\Large\bfseries]{section}

\RedeclareSectionCommand[
beforeskip=1em plus 0.2em minus 0.2em,
afterskip=0.5em,
font=\normalfont\large\bfseries]{subsection}

% ---------------------------------------------------------
% TITRE DU RAPPORT
% ---------------------------------------------------------
\title{\Huge \bfseries Projet de synthèse AVI}

\author{%
	\Large Marius Michaud\\
	\texttt{marius.michaud@alumni.enac.fr}\\[1.2em]
	\Large Guillaume Claudon\\
	\texttt{guillaume.claudon@alumni.enac.fr}\\[1.2em]
	\Large Ulysse Bourdic-Girard\\
	\texttt{ulysse.bourdic-girard@alumni.enac.fr}\\[2em]
	\large École Nationale de l’Aviation Civile (ENAC)%
}

\date{Janvier 2026}

\publishers{
	\includegraphics[width=5cm]{logo_enac.png}
}


% ---------------------------------------------------------
\begin{document}
	% WARN: Le document ne doit pas faire plus de 20 pages!
	
	% ---------------------------------------------------------
	\maketitle
	\tableofcontents
	\nopagebreak
	\listoffigures
	\listoftables
	\clearpage
	\bibliographystyle{plain-fr}
	\bibliography{rapportCAdV}
	%----------------------------------------------------------
	
	%----------------------------------------------------------
	\chapter{Introduction}
	%----------------------------------------------------------
	Le but de ce projet est d'implémenter une loi de descente pour un avion de ligne. Cet avion est en vol stabilisé à un niveau de vol (FL) et une vitesse équivalente (\emph{EAS}) qui dépendent de chaque groupe. Il est autorisé par le contrôle aérien à descendre jusqu'au niveau $30$, où il se stabilise à une vitesse de \SI{230}{kts} \cite[4.1]{Sujet}. Le comportement de cette loi est synthétisé dans le schéma suivant:
	
	\begin{figure}[H]
		\centering
		\includegraphics[width=\textwidth]{schema_sujet.png}
		\caption{Comportement de la loi de descente \cite[Fig 4.1]{Sujet}}
		\label{fig:schema_sujet}
	\end{figure}
	
	Dans le cadre de ce projet, on ne considère que l'axe longitudinal de l'avion, on néglige l'influence des commandes de roulis et de lacet. Le modèle de l'avion est donc commandé par trois entrées \cite[4.3]{Sujet}:
	\begin{itemize}
		\item $\delta_{\text{PHR}}$: le braquage du plan horizontal réglable (PHR)
		\item $\delta_{el}$: le braquage de la gouverne de profondeur
		\item $\delta_{\text{THR}}$: la position de la manette de poussée
	\end{itemize}
	
	Pour remplir ce cahier des charges, nous allons découper la commande de l'avion en trois sous-commandes distinctes:
	\begin{itemize}
		\item \textbf{La commande de l'assiette} pour laquelle on implémentera une commande classique ;
		\item \textbf{La loi de descente} pour laquelle on implémentera une commande modale ;
		\item \textbf{Les lois de maintien du niveau et de la vitesse} pour lesquelles on implémentera une commande optimale.
	\end{itemize}
	
	%----------------------------------------------------------
	\chapter{Modélisation et Linéarisation}
	%----------------------------------------------------------
	Avant de procéder à la synthèse des lois de commande, il est nécessaire de définir le modèle de l'avion et son point de fonctionnement.
	
	\section{Caractéristiques de l'avion}
	L'avion étudié est un Airbus A320 (groupe Golf). Il est caractérisé par sa marge statique (\emph{ms}) et son coefficient de masse (\emph{km}). Le tableau suivant récapitule les paramètres de simulation :
	
	\begin{table}[H]
		\centering	
		\caption{Paramètres caractéristiques de l'A320 (Groupe Golf) \cite[Tab 2.1]{Sujet}}
		\label{tab:parametres}
		\begin{tabular}{l l}
			\toprule
			\textbf{Paramètre} & \textbf{Valeur} \\
			\midrule
			Marge statique (\emph{ms}) & $0.5$ \\
			Coefficient de masse (\emph{km}) & $0.3$ \\
			Niveau de vol initial (FL) & $150$ \\
			Vitesse équivalente (\emph{EAS}) & \SI{260}{kts} \\
			\bottomrule
		\end{tabular}
	\end{table}
	
	\section{Point de fonctionnement (Trim)}
	Conformément au scénario \textit{Open Descent} décrit dans le sujet \cite{Sujet}, la poussée est maintenue fixe sur la position \emph{Idle}. Le calcul du point d'équilibre (\textit{trim}) a été réalisé via la fonction \texttt{utComputeTrimIdle}. Pour l'avion assigné à l'équipe Golf, le point d'équilibre obtenu est :
	\begin{itemize}
		\item Altitude : \SI{15000}{ft} (\SI{4572}{m})
		\item Vitesse vraie ($V_a$) : $\approx$ \SI{139}{m/s} (pour \SI{260}{kts} EAS)
		\item Poussée : $\delta_{\text{THR}} = 0.1$ (Idle)
		\item Assiette ($\theta_{trim}$) et Incidence ($\alpha_{trim}$) : calculées pour assurer la portance.
	\end{itemize}
	
	\section{Linéarisation}
	La linéarisation du modèle non-linéaire autour de ce point de trim a été effectuée à l'aide de la commande \texttt{linmod} de Matlab. Cela nous a permis d'extraire le modèle d'état longitudinal réduit (matrices $A_4$, $B_4$) décrivant la dynamique des états $[V_a, \alpha, \theta, q]^T$ sous l'action de la gouverne de profondeur $\delta_{el}$.
	
	L'analyse des valeurs propres de la matrice $A_4$ (obtenues via la commande \texttt{damp(A4)}) met en évidence les deux modes longitudinaux classiques :
	
	\begin{itemize}
		\item \textbf{Le mode phugoïde} : mode lent caractérisé par une pulsation naturelle $\omega_n = \SI{0.0721}{rad/s}$ et un amortissement faible $\zeta = 0.1267$.
		\item \textbf{L'oscillation d'incidence} : mode rapide caractérisé par une pulsation naturelle $\omega_n = \SI{2.8896}{rad/s}$ et un amortissement plus important $\zeta = 0.5167$.
	\end{itemize}
	
	Ces résultats sont résumés dans le tableau suivant :
	
	\begin{table}[H]
		\centering
		\caption{Analyse modale du modèle longitudinal linéarisé au trim}
		\label{tab:modes_propres}
		\begin{tabular}{l S S}
			\toprule
			\textbf{Mode} & {\textbf{Amortissement} ($\zeta$)} & {\textbf{Pulsation} ($\omega_n$ [\si{rad/s}])} \\
			\midrule
			Phugoïde & 0.1267 & 0.0721 \\
			Oscillation d'incidence & 0.5167 & 2.8896 \\
			\bottomrule
		\end{tabular}
	\end{table}
	
	
	%----------------------------------------------------------
	\chapter{Commande de l'assiette (Boucle Interne)}
	%----------------------------------------------------------
	
	Ce chapitre détaille la conception de la boucle interne de pilotage, dont l'objectif est d'asservir l'assiette longitudinale $\theta$ de l'avion via la gouverne de profondeur $\delta_{el}$ \cite[4.5]{Sujet}.
	
	\section{Synthèse du correcteur}
Nous avons retenu une structure de commande classique de type PI-D (Proportionnel Intégral, Dérivée sur la mesure) associée à un pré-filtre sur la consigne. Les gains ont été calculés via la méthode de Wang, présentée dans le sujet au paragraphe 3.5.1 du sujet \cite{Sujet} et dans l'article de \emph{Wang et al.} \cite{Wang2009}.
	
	Conformément au cahier des charges, nous avons conservé les spécifications initiales pour la méthode de Wang, à savoir :
	\begin{equation}
		t_s = \SI{3}{s} \quad \text{et} \quad D = 5\%
	\end{equation}
	
	Ce réglage permet de placer les pôles dominants de la boucle fermée interne afin de respecter les critères de performance temporelle et de dépassement demandés.
	
	\section{Implémentation et Validation}
	
	Les structures de commande ont été implémentées sous Simulink pour le modèle linéarisé (servant à la synthèse) et le modèle non-linéaire complet (servant à la validation).
	
	\begin{figure}[htbp] 
		\centering
		\begin{subfigure}[b]{0.85\textwidth}
			\centering
			\includegraphics[width=\textwidth]{schema_lineaire_assiette.png}
			\caption{Système linéarisé}
		\end{subfigure}
		
		\par\bigskip 
		
		\begin{subfigure}[b]{0.85\textwidth}
			\centering
			\includegraphics[width=\textwidth]{schema_classique.png}
			\caption{Système non-linéaire (Avion naturel)}
		\end{subfigure}
		
		\vspace{-0.5em} 
		\caption{Schémas Simulink de la boucle de tenue d'assiette}
		\label{fig:schemas_assiette}
	\end{figure}
	
	Afin de valider la synthèse et la qualité de la linéarisation, nous avons superposé la réponse indicielle des deux modèles pour une consigne en échelon d'amplitude \SI{5}{\degree} partant du point de trim.
	
	\begin{figure}[htbp]
		\centering
		\makebox[\textwidth][c]{\includegraphics[trim={0 0 0 0.82cm}, clip, width=1.1\textwidth]{reponse_bilineaire_assiette.png}}
		\vspace{-1em} 
		\caption{Comparaison des réponses indicielles (Linéaire vs Non-linéaire)}
		\label{fig:reponse_bilineaire_assiette}
	\end{figure}
	
	L'analyse de la figure \ref{fig:reponse_bilineaire_assiette} suggère deux commentaires majeurs :
	\begin{enumerate}
		\item \textbf{Respect des spécifications :} Le temps de réponse à 5\% est bien de l'ordre de 3 secondes et le dépassement correspond aux 5\% attendus (pic à $\approx \SI{5.25}{\degree}$).
		\item \textbf{Validité du modèle :} Les courbes rouge (modèle linéaire) et bleue (modèle non-linéaire) sont parfaitement superposées. Cela confirme que pour des amplitudes faibles (voisines de 5°), la linéarisation autour du point de trim est une excellente approximation de la dynamique réelle de l'avion.
	\end{enumerate}
	
	
		
	%----------------------------------------------------------
	\chapter{Loi de descente (Open Descent)}
	%----------------------------------------------------------
	
	L'objectif de la loi de descente est de maintenir une vitesse équivalente (EAS) constante en agissant sur la consigne d'assiette $\theta_c$, la poussée restant figée à \emph{Idle} \cite[4.6]{Sujet}.
	
	\section{Stratégie de commande modale}
	La linéarisation est effectuée sur le modèle Simulink \texttt{acDynModelGolfTenueClassique}, qui inclut déjà la boucle interne d'assiette. Le système linéarisé $A_6$ est d'ordre 6 et correspond au vecteur d'état suivant :
	\begin{equation}
		x = [V_a, \alpha, \theta, q, x_{\text{pid}}, x_{\text{pf}}]^T
	\end{equation}
	Où $x_{\text{pid}}$ et $x_{\text{pf}}$ sont les états internes du correcteur PI-D et du pré-filtre de la boucle interne.
	La commandabilité de ce système augmenté a été vérifiée via le rang de la matrice de commandabilité (fonction MatLab \texttt{ctrb}).

	La commande modale consite à choisir les pôles du système bouclé pour influencer le comportement de ses modes. La fonction MatLab \texttt{place} permet de calculer le gain du retour d'état qui sera utilisé pour commander le système (on note ce gain $K_x$).

	Le retour d'état seul ne permet pas de commander le système, on ajoute une précommande pour adapter la commande au retour d'état. La précommande est calculée ainsi \cite[eq. 3.50]{Sujet}:

	\begin{equation}
		H = -(C \cdot (A-B K)^{-1}\cdot B)^{-1} 
	\end{equation}
	\pagebreak

	L'architecture de commande du système est illustrée ci-dessous:

	\begin{figure}[H]
		\centering
		\makebox[\textwidth][c]{\includegraphics[width=1.2\textwidth]{schema_lineaire_modale_prefiltre.png}}
		\caption{Schéma Simulink de la commande avec préfiltre $H$}
		\label{fig:modale_prefiltre}
	\end{figure}

	\section{Observateur d'état}
	Pour implémenter cette loi de commande par retour d'état, nous devons connaître l'ensemble du vecteur d'état $x$. Or, certaines grandeurs comme l'incidence $\alpha$ ne sont pas toujours mesurées directement ou sont bruitées.
	
	Pour reconstruire le vecteur d'état complet, nous utilisons un observateur de Luenberger défini par :
	\begin{equation}
		\dot{\hat{x}} = A \hat{x} + B u + L (y_{\text{mes}} - C \hat{x})
	\end{equation}
	Où $\hat{x}$ est l'état estimé et $L$ la matrice de gain de l'observateur. Cet observateur est synthétisé en utilisant les trois sorties mesurées disponibles définies dans la matrice $C_{\text{obs}}$ :
	\begin{equation}
		y_{mes} = [V_a, \theta, q]^T
	\end{equation}
	L'observabilité de la paire $(A_6, C_{\text{obs}})$ a été validée (rang plein).
	
	Pour le calcul du gain $L$, bien que la théorie suggère généralement une dynamique d'erreur d'estimation plus rapide que celle de la commande, nous avons choisi de placer les pôles de l'observateur sur les mêmes dynamiques que celles du système en boucle fermée (vecteur \texttt{poles\_desires(1:6)}). Ce choix vise à limiter l'amplification du bruit de mesure sur les états estimés, notamment sur l'incidence $\alpha$. D'après les tests effectués, ce réglage s'est avéré satisfaisant pour notre application et une dynamique plus rapide n'a pas semblé nécessaire.

	\section{Choix des pôles}
	Pour calculer le gain du retour d'état, nous avons choisi les pôles suivants :
	\begin{itemize}
		\item \textbf{Dynamique dominante ($dp$)} : Conservée de la méthode de Wang (boucle interne).
		\item \textbf{Modes rapides} : Pôles complexes conjugués conservés vers $-3.4 \pm 0.15i$ (observés dans la matrice $A_6$ via la fonction MatLab \texttt{eig}). Nous avons choisi de ne pas modifier ces modes naturels pour éviter de solliciter inutilement les actionneurs.
		\item \textbf{Pôles réels de réglage} : Fixés à $-3$ et $-4$.
	\end{itemize}

	La simulation de la commande modale donne le résultat suivant:
	
	\begin{figure}[H]
		\centering
		\makebox[\textwidth][c]{\includegraphics[width=\textwidth]{reponse_lineaire_modale_prefiltre.png}}
		\caption{Réponse de la commande modale à une consigne en échelon d'amplitude 3}
		\label{fig:reponse_modale_prefiltre}
	\end{figure}

% TODO: Détailler
	La réponse de cette commande n'est pas du tout satisfaisante: l'erreur statique est de l'ordre de $2.42$.

	\section{Gestion de l'erreur statique : Ajout d'un intégrateur}
	Pour pallier les limitations de la précommande seule, le sujet suggère l'ajout d'un intégrateur pour annuler l'erreur statique de vitesse \cite[3.6.2]{Sujet}. Cette approche est indispensable car le modèle linéaire varie fortement au cours de la descente.
	Nous avons donc augmenté le système avec un état intégrateur $\xi$ défini par :
	\begin{equation}
		\dot{\xi} = V_{\text{EAS,cons}} - V_{\text{EAS,mes}}
	\end{equation}

	On choisit pour l'intégrateur un pôle à $-10.33$, ce qui assure une élimination rapide de l'erreur statique, et vérifie la contrainte de stabilité imposée ($\text{Re} < -1.3$).

	Le schéma de la commande modale avec intégrateur appliquée au système linéaire est illustrée ci-dessous:

	\begin{figure}[H]
		\centering
		\makebox[\textwidth][c]{\includegraphics[width=1.2\textwidth]{schema_lineaire_modale.png}}
		\caption{Schéma Simulink de la boucle de vitesse avec intégrateur sur le système linéaire}
		\label{fig:schema_lineaire_modale}
	\end{figure}
	
	La simulation de la commande modale avec intégrateur donne le résultat suivants:
	\begin{figure}[H]
		\centering
		\makebox[\textwidth][c]{\includegraphics[width=\textwidth]{reponse_lineaire_modale_integrateur.png}}
		\caption{Réponse de la commande modale à une consigne en échelon d'amplitude 3}
		\label{fig:reponse_modale_integrateur}
	\end{figure}

	Cette réponse est satisafaisante: le système répond en environ \SI{3.5}{s} et avec un dépassement de l'ordre de \SI{6}{\%}. Nous avons donc implémenté cette commande sur le modèle non linéaire.

	\section{Modèle non linéaire}

	Le schéma suivant illustre l'intégration de la commande modale sur le système non linéaire. Nous avons conservé la possibilité d'alterner entre la précomande $H$ et l'intégrateur, même si l'intégrateur se révèle être supérieur. Il est aussi important de noter que l'intégrateur agit sur l'erreur d'EAS (et non de TAS), car c'est la grandeur pilotée.

	\begin{figure}[H]
		\centering
		\makebox[\textwidth][c]{\includegraphics[width=\textwidth]{schema_nonlineaire_modale.png}}
		\caption{Schéma de la commande modale appliquée au modèle non linéaire}
		\label{fig:schema_modale_nonlineaire}
	\end{figure}
	
	\section{Résultats de simulation}
	La validation sur le modèle non-linéaire (sur une durée de \SI{600}{s}) démontre le bon fonctionnement de la loi. On observe sur la figure \ref{fig:resultat_vitesse} que la Vitesse Vraie (\emph{TAS}) diminue progressivement au cours de la descente pour maintenir une \emph{EAS} constante, validant la physique du vol.

% TODO: Add sim results	

	\begin{figure}[H]
		\centering
		%\includegraphics[width=0.8\textwidth]{resultat_vitesse.png} % Placeholder
		\caption{Évolution de la Vitesse Vraie (TAS) lors de la descente à EAS constante}
		\label{fig:resultat_vitesse}
	\end{figure}
	
	%----------------------------------------------------------
	\chapter{Lois de maintien du niveau et de la vitesse}
	%----------------------------------------------------------
	
	Pour la phase de stabilisation au niveau 30, l'avion doit maintenir simultanément son altitude et sa vitesse aérodynamique de \SI{230}{kts}. Il s'agit d'un problème multi-variable (MIMO) où la poussée $\delta_{thr}$ et la profondeur $\delta_{el}$ sont utilisées conjointement.

	\section{Linéarisation}

	Pour metttre en œuvre la commande, on linéarise le modèle avion au nouveau point de trim: \SI{230}{kts} et \SI{3000}{ft}. Cette linéarisation est réalisée par \texttt{utAcDynamicsFunction.m} et donne le point de fonctionnement suivant: 
	\begin{equation}
		\text{trimValLF} = \left( \begin{array}{c}
				\delta_{\text{PHR}} \\
				\alpha_{\text{trim}} \\
				\delta_{\text{THR}}
		\end{array} \right) = 
		\left( \begin{array}{c}
				\SI{-10.0}{\degree} \\
				\SI{3.6}{\degree} \\
				\SI{0.25}{\%}	
		\end{array} \right)
	\end{equation}

	\section{Synthèse par commande optimale (LQT)}
	Nous avons implémenté une structure \textit{Linear Quadratic Tracker} (LQT) pour minimiser l'erreur de suivi entre la sortie du système $y$ et un modèle de référence $y_r$. Le critère à minimiser est défini par :
	\begin{equation}
		J = \frac{1}{2} \int_{0}^{\infty} (e^T Q e + \tilde{u}^T R \tilde{u}) dt
	\end{equation}
	Où $Q$ et $R$ sont les matrices de pondération permettant de régler le compromis entre précision du suivi et énergie de commande.
	
	\section{Réglage des matrices de pondération}
	Pour respecter les spécifications (temps de réponse $< \SI{20}{s}$ pour $V_a$ et $< \SI{2}{s}$ pour la pente, sans dépassement), nous avons choisi :
	\begin{itemize}
		\item \textbf{Matrice $Q$} : Une pondération forte a été placée sur l'erreur de pente pour garantir la réactivité de la boucle d'altitude.
		\item \textbf{Matrice $R$} : Les coefficients ont été ajustés pour éviter des sollicitations brutales des moteurs (limitation de la fatigue) et des gouvernes.
	\end{itemize}
	
	
	
	\section{Analyse du couplage}
	Contrairement aux lois SISO précédentes, la commande optimale LQT permet un \textbf{découplage} partiel des axes. Une variation de consigne de vitesse est compensée par une action combinée sur la poussée (action directe) et sur la profondeur (pour contrer le moment cabreur induit par les moteurs).
	
	
	%----------------------------------------------------------
	\chapter{Simulation du scénario complet}
	%----------------------------------------------------------
	
	\section{Gestion de la transition}
	Le passage de la loi \textit{Open Descent} (vitesse tenue par l'assiette) aux lois de maintien (vitesse tenue par la poussée) constitue une phase critique du vol. La transition est déclenchée lors de la capture de l'altitude cible (\SI{3000}{ft}). 
	
	\section{Analyse des résultats globaux}
	La simulation complète montre une descente fluide à \SI{260}{kts} EAS. Lors de l'approche du FL30 :
	\begin{enumerate}
		\item La loi de capture d'altitude réduit la pente.
		\item La manette des gaz sort de la position \textit{Idle} pour maintenir la vitesse.
		\item L'avion se stabilise aux nouvelles consignes sans dépassement excessif, validant la robustesse des correcteurs sur le modèle non-linéaire.
	\end{enumerate}
	
	%----------------------------------------------------------
	\chapter{Conclusion}
	%----------------------------------------------------------
	
	Ce projet a permis de valider une architecture de pilotage automatique complète. Nous avons pu observer que :
	\begin{itemize}
		\item La linéarisation est un outil puissant mais nécessite l'ajout d'intégrateurs pour compenser les dérives du modèle non-linéaire en cours de descente.
		\item La commande optimale MIMO offre une bien meilleure gestion des interactions entre les axes que la simple juxtaposition de boucles SISO.
	\end{itemize}
	
	\paragraph{Limites et perspectives} 
	Pour améliorer le réalisme, il serait nécessaire d'intégrer des saturations sur les vitesses de déflexion des gouvernes (\SI{15}{\degree/s} pour la profondeur) et de prendre en compte la dynamique plus lente du PHR par rapport à la gouverne de profondeur.
\end{document}

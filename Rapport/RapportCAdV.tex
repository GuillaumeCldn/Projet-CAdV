\documentclass[12pt,oneside]{scrreprt}

% ---------------------------------------------------------
% PACKAGES
% ---------------------------------------------------------
\usepackage[utf8]{inputenc}
\usepackage[T1]{fontenc}
\usepackage[french]{babel}
\usepackage{graphicx}
\usepackage{subcaption}
\usepackage{amsmath}
\usepackage{siunitx}
\sisetup{output-decimal-marker = {,}}
\usepackage{booktabs}
\usepackage[hidelinks]{hyperref}
\usepackage{lmodern}
\usepackage{microtype}
\usepackage{float}
\usepackage{fmtcount}
\usepackage{enumitem}
\usepackage{scrhack}
\usepackage{tikz}
\usetikzlibrary{arrows.meta, matrix, decorations.pathreplacing, calc}
\usepackage{wrapfig}

% ---------------------------------------------------------
% MISE EN PAGE
% ---------------------------------------------------------
\usepackage[margin=2.5cm]{geometry}

\setlength{\parskip}{0.85em}
\setlength{\parindent}{1.2em}
\linespread{1.15}

% ---------------------------------------------------------
% TITRES (méthode KOMA-Script)
% ---------------------------------------------------------
\RedeclareSectionCommand[
beforeskip=1.8em plus 0.2em minus 0.2em,
afterskip=1em,
font=\normalfont\huge\bfseries]{chapter}

\RedeclareSectionCommand[
beforeskip=1.4em plus 0.2em minus 0.2em,
afterskip=0.8em,
font=\normalfont\Large\bfseries]{section}

\RedeclareSectionCommand[
beforeskip=1em plus 0.2em minus 0.2em,
afterskip=0.5em,
font=\normalfont\large\bfseries]{subsection}

% ---------------------------------------------------------
% TITRE DU RAPPORT
% ---------------------------------------------------------
\title{\Huge \bfseries Projet de synthèse AVI}

\author{%
	\Large Marius Michaud\\
	\texttt{marius.michaud@alumni.enac.fr}\\[1.2em]
	\Large Guillaume Claudon\\
	\texttt{guillaume.claudon@alumni.enac.fr}\\[1.2em]
	\Large Ulysse Bourdic-Girard\\
	\texttt{ulysse.bourdic-girard@alumni.enac.fr}\\[2em]
	\large École Nationale de l’Aviation Civile (ENAC)%
}

\date{Janvier 2026}

\publishers{
	\includegraphics[width=5cm]{logo_enac.png}
}


% ---------------------------------------------------------
\begin{document}
	% WARN: Le document ne doit pas faire plus de 20 pages!
	
	% ---------------------------------------------------------
	% ---------------------------------------------------------
	\maketitle
	\tableofcontents
	
	\newpage
	\begingroup
	
	\let\clearpage\relax
	\let\cleardoublepage\relax
	
	\KOMAoptions{listof=leveldown, bibliography=leveldown}
	
	\listoffigures
	\vspace{1cm} % Ajuster cette valeur selon le remplissage de la page
	
	
	\bibliographystyle{plain-fr}
	\bibliography{rapportCAdV}
	\endgroup
	\clearpage
	
	%----------------------------------------------------------
	
	%----------------------------------------------------------
	\chapter{Introduction}
	%----------------------------------------------------------
	Le but de ce projet est d'implémenter une loi de descente pour un avion de ligne. Cet avion est en vol stabilisé à un niveau de vol (FL) et une vitesse équivalente (\emph{EAS}) qui dépendent de chaque groupe. Il est autorisé par le contrôle aérien à descendre jusqu'au niveau $30$, où il se stabilise à une vitesse de \SI{230}{kts}~\cite[4.1]{Sujet}. Le comportement de cette loi est synthétisé dans le schéma suivant :
	
	\begin{figure}[H]
		\centering
		\includegraphics[width=\textwidth]{schema_sujet.png}
		\caption{Comportement de la loi de descente~\cite[Fig 4.1]{Sujet}}
		\label{fig:schema_sujet}
	\end{figure}
	
	Dans le cadre de ce projet, on ne considère que l'axe longitudinal de l'avion, on néglige l'influence des commandes de roulis et de lacet. Le modèle de l'avion est donc commandé par trois entrées~\cite[4.3]{Sujet} :
	\begin{itemize}
		\item $\delta_{\text{PHR}}$ : le braquage du plan horizontal réglable (PHR)
		\item $\delta_{el}$ : le braquage de la gouverne de profondeur
		\item $\delta_{\text{THR}}$ : la position de la manette de poussée
	\end{itemize}
	
	Pour remplir ce cahier des charges, nous allons découper la commande de l'avion en trois sous-commandes distinctes :
	\begin{itemize}
		\item \textbf{La commande de l'assiette} pour laquelle on implémentera une commande classique ;
		\item \textbf{La loi de descente} pour laquelle on implémentera une commande modale ;
		\item \textbf{Les lois de maintien du niveau et de la vitesse} pour lesquelles on implémentera une commande optimale.
	\end{itemize}
	
	%----------------------------------------------------------
	\chapter{Modélisation et Linéarisation}
	%----------------------------------------------------------
	Avant de procéder à la synthèse des lois de commande, il est nécessaire de définir le modèle de l'avion et son point de fonctionnement.
	
	\section{Caractéristiques de l'avion}
	L'avion étudié est un Airbus A320 (groupe Golf). Il est caractérisé par sa marge statique (\emph{ms}) et son coefficient de masse (\emph{km}). Le tableau suivant récapitule les paramètres de simulation :
	
	\begin{table}[H]
		\centering	
		\caption{Paramètres caractéristiques de l'A320 (Groupe Golf)~\cite[Tab 2.1]{Sujet}}
		\label{tab:parametres}
		\begin{tabular}{l l}
			\toprule
			\textbf{Paramètre} & \textbf{Valeur} \\
			\midrule
			Marge statique (\emph{ms}) & $0.5$ \\
			Coefficient de masse (\emph{km}) & $0.3$ \\
			Niveau de vol initial (FL) & $150$ \\
			Vitesse équivalente (\emph{EAS}) & \SI{260}{kts} \\
			\bottomrule
		\end{tabular}
	\end{table}
	
	\section{Point de fonctionnement (Trim)}
	Conformément au scénario \textit{Open Descent} décrit dans le sujet~\cite{Sujet}, la poussée est maintenue fixe sur la position \emph{Idle}. Le calcul du point d'équilibre (\textit{trim}) a été réalisé via la fonction \texttt{utComputeTrimIdle}. Pour l'avion assigné à l'équipe Golf, le point d'équilibre obtenu est :
	\begin{itemize}
		\item Altitude : \SI{15000}{ft} (\SI{4572}{m})
		\item Vitesse vraie ($V_a$) : $\approx$ \SI{139}{m/s} (pour \SI{260}{kts} EAS)
		\item Poussée : $\delta_{\text{THR}} = 0.1$ (Idle)
		\item Assiette ($\theta_{trim}$) et Incidence ($\alpha_{trim}$) : calculées pour assurer la portance.
	\end{itemize}
	
	\section{Linéarisation}
	La linéarisation du modèle non-linéaire autour de ce point de trim a été effectuée à l'aide de la commande \texttt{linmod} de Matlab. Cela nous a permis d'extraire le modèle d'état longitudinal réduit (matrices $A_4$, $B_4$) décrivant la dynamique des états $[V_a, \alpha, \theta, q]^T$ sous l'action de la gouverne de profondeur $\delta_{el}$.
	
	L'analyse des valeurs propres de la matrice $A_4$ (obtenues via la commande \texttt{damp(A4)}) met en évidence les deux modes longitudinaux classiques :
	
	\begin{itemize}
		\item \textbf{Le mode phugoïde} : mode lent caractérisé par une pulsation naturelle $\omega_n = \SI{0.0721}{rad/s}$ et un amortissement faible $\zeta = 0.1267$.
		\item \textbf{L'oscillation d'incidence} : mode rapide caractérisé par une pulsation naturelle $\omega_n = \SI{2.8896}{rad/s}$ et un amortissement plus important $\zeta = 0.5167$.
	\end{itemize}
	
	Ces résultats sont résumés dans le tableau suivant :
	
	\begin{table}[H]
		\centering
		\caption{Analyse modale du modèle longitudinal linéarisé au trim}
		\label{tab:modes_propres}
		\begin{tabular}{l S S}
			\toprule
			\textbf{Mode} & {\textbf{Amortissement} ($\zeta$)} & {\textbf{Pulsation} ($\omega_n$ [\si{rad/s}])} \\
			\midrule
			Phugoïde & 0.1267 & 0.0721 \\
			Oscillation d'incidence & 0.5167 & 2.8896 \\
			\bottomrule
		\end{tabular}
	\end{table}
	
	
	%----------------------------------------------------------
	\chapter{Commande de l'assiette (Boucle Interne)}
	%----------------------------------------------------------
	
	Ce chapitre détaille la conception de la boucle interne de pilotage, dont l'objectif est d'asservir l'assiette longitudinale $\theta$ de l'avion via la gouverne de profondeur $\delta_{el}$~\cite[4.5]{Sujet}.
	
	\section{Synthèse du correcteur}
	
	\noindent
	\begin{minipage}{0.60\textwidth}
		Nous avons retenu une structure de commande classique de type PI-D (Proportionnel, Intégral, Dérivé sur la mesure) associée à un pré-filtre sur la consigne. Les gains ont été calculés via la méthode de Wang, présentée dans le sujet au paragraphe 3.5.1~\cite{Sujet} et dans l'article de \emph{Wang et al.}~\cite{Wang2009}. 
		
		La méthode de Wang demande le placement d'un pôle supplémentaire pour calculer les gains du PI-D. Nous avons choisi celui-ci avec un graphique \og root-locus\fg{} (lieu des racines) en cherchant un pôle qui est plus rapide que les actuels afin de ne pas affecter la dynamique du système. On a choisi un $K_d$ par cette méthode à -5.66, qui permet de maximiser l'efficacité du système, en plaçant des pôles très proches et rapides, en évitant la divergence des branches du root locus.
	\end{minipage}
	\hfill 
	\begin{minipage}{0.35\textwidth}
		\centering
		\includegraphics[width=\linewidth]{rootlocus_Wang_classique.png}
		\captionof{figure}{Lieu des racines}
		\label{fig:wang_rootlocus}
	\end{minipage}
	
	Conformément au cahier des charges, nous avons conservé les spécifications initiales pour la méthode de Wang, à savoir :
	\begin{equation}
		t_s = \SI{3}{s} \quad \text{et} \quad D = 5\%
	\end{equation}
	
	Ce réglage permet de placer les pôles dominants de la boucle fermée interne afin de respecter les critères de performance temporelle et de dépassement demandés.
	
	\section{Implémentation et Validation}
	
	Les structures de commande ont été implémentées sous Simulink pour le modèle linéarisé (servant à la synthèse) et le modèle non-linéaire complet (servant à la validation).
	
	\begin{figure}[htbp]
		\centering
		\begin{subfigure}[b]{0.85\textwidth} 
			\centering
			\includegraphics[width=\textwidth]{schema_lineaire_assiette.png}
			\caption{Système linéarisé}
		\end{subfigure}
		
		\par\bigskip 
		
		\begin{subfigure}[b]{0.85\textwidth}
			\centering
			\includegraphics[width=\textwidth]{schema_classique.png}
			\caption{Système non-linéaire (Avion naturel)}
		\end{subfigure}
		
		\vspace{-0.5em}
		\caption{Schémas Simulink de la boucle de tenue d'assiette}
		\label{fig:schemas_assiette}
	\end{figure}
	
	\vspace{0.5cm}
	Afin de valider la synthèse, nous avons étudié la réponse du système pour un échelon en consigne de \SI{5}{\degree}.
	
	\begin{figure}[htbp]
		\centering
		\begin{minipage}[c]{0.58\textwidth}
			\centering
			\includegraphics[trim={0 0 0 0.82cm}, clip, width=\linewidth]{reponse_lineaire_assiette.png}
			\caption{Réponse du système linéarisé à un échelon en consigne}
			\label{fig:reponse_bilineaire_assiette}
		\end{minipage}
		\hfill
		\begin{minipage}[c]{0.38\textwidth}
			\small 
			\raggedright
			L'analyse de la figure~\ref{fig:reponse_bilineaire_assiette} suggère :
			
			\begin{enumerate}[leftmargin=*] 
				\item \textbf{Spécifications :} Temps de réponse $\approx \SI{3}{s}$, dépassement $\approx 5\%$ (pic à $\approx \SI{5.25}{\degree}$).
				\item \textbf{Validité :} On constate que la courbe est identique au cas non-linéaire par la suite. La linéarisation est valide pour les faibles amplitudes ($5^\circ$).
			\end{enumerate}
		\end{minipage}
	\end{figure}
	
	
	
	%----------------------------------------------------------
	\chapter{Loi de descente (Open Descent)}
	%----------------------------------------------------------
	
	L'objectif de la loi de descente est de maintenir une vitesse équivalente (EAS) constante en agissant sur la consigne d'assiette $\theta_c$, la poussée restant figée à \emph{Idle}~\cite[4.6]{Sujet}.
	
	\section{Stratégie de commande modale}
	La linéarisation est effectuée sur le modèle Simulink \texttt{Golf\_Lin\_Modale}, qui inclut déjà la boucle interne d'assiette. Le système linéarisé $A_6$ est d'ordre 6 et correspond au vecteur d'état suivant :
	\begin{equation}
		x = [V_a, \alpha, \theta, q, x_{\text{pid}}, x_{\text{pf}}]^T
	\end{equation}
	Où $x_{\text{pid}}$ et $x_{\text{pf}}$ sont les états internes du correcteur PI-D et du pré-filtre de la boucle interne.
	La commandabilité de ce système augmenté a été vérifiée via le rang de la matrice de commandabilité (fonction MatLab \texttt{ctrb}).
	
	La commande modale consiste à choisir les pôles du système bouclé pour influencer le comportement de ses modes. La fonction MatLab \texttt{place} permet de calculer le gain du retour d'état qui sera utilisé pour commander le système (on note ce gain $K_x$).
	
	Le retour d'état seul ne permet pas de commander le système, car celui-ci présente une erreur statique en l'absence d'une précommande. Cette dernière est calculée comme suit~\cite[eq. 3.50]{Sujet}:
	
	\begin{equation}
		H = -(C \cdot (A-B K)^{-1}\cdot B)^{-1}
	\end{equation}
	\pagebreak
	
	L'architecture de commande du système est illustrée ci-dessous :
	
	\begin{figure}[H]
		\centering
		\makebox[\textwidth][c]{\includegraphics[width=1.2\textwidth]{schema_lineaire_modale_prefiltre.png}}
		\caption{Schéma Simulink de la commande avec préfiltre $H$}
		\label{fig:modale_prefiltre}
	\end{figure}
	
	\section{Observateur d'état}
	Pour implémenter cette loi de commande par retour d'état, nous devons connaître l'ensemble du vecteur d'état $x$. Or, certaines grandeurs comme l'incidence $\alpha$ ne sont pas toujours mesurées directement ou sont bruitées.
	
	Pour reconstruire le vecteur d'état complet, nous utilisons un observateur de Luenberger défini par :
	\begin{equation}
		\dot{\hat{x}} = A \hat{x} + B u + L (y_{\text{mes}} - C \hat{x})
	\end{equation}
	Où $\hat{x}$ est l'état estimé et $L$ la matrice de gain de l'observateur. Cet observateur est synthétisé en utilisant les trois sorties mesurées disponibles définies dans la matrice $C_{\text{obs}}$ :
	\begin{equation}
		y_{mes} = [V_a, \theta, q]^T
	\end{equation}
	L'observabilité de la paire $(A_6, C_{\text{obs}})$ a été validée (rang plein).
	
	Pour le calcul du gain $L$, afin de respecter la dynamique d'erreur d'estimation qui doit être plus rapide que celle de la commande, nous avons choisi de placer les pôles de l'observateur à trois fois la dynamique du système en boucle fermée (soit $3 \times$ \texttt{poles\_desires(1:6)}). Ce réglage nous semble être le compromis le plus sain : il assure une convergence suffisamment rapide de l'estimation pour le contrôle, tout en évitant une dynamique excessive qui amplifierait inutilement le bruit de mesure sur les états estimés, notamment sur l'incidence $\alpha$. On a validé la convergence rapide de l'observateur d'état en mesurant l'erreur d'estimation dans le cas d'une condition initiale randomisée, que l'observateur était chargé de filtrer.
	
	\section{Choix des pôles}
	Pour calculer le gain du retour d'état, nous avons choisi les pôles suivants :
	\begin{itemize}
		\item \textbf{Dynamique dominante ($dp$)} : Conservée de la méthode de Wang (boucle interne).
		\item \textbf{Modes rapides} : Pôles complexes conjugués conservés vers $-3.4 \pm 0.15i$ (observés dans la matrice $A_6$ via la fonction MatLab \texttt{eig}). Nous avons choisi de ne pas modifier ces modes naturels pour éviter de solliciter inutilement les actionneurs.
		\item \textbf{Pôles réels de réglage} : Fixés à $-3$ et $-4$.
	\end{itemize}
	
	La simulation de la commande modale donne le résultat suivant :
	
	\begin{figure}[H]
		\centering
		\makebox[\textwidth][c]{\includegraphics[width=\textwidth]{reponse_comparee_modale_lineaire_legendee.png}}
		\caption{Réponse de la commande modale à une consigne en échelon d'amplitude 3}
		\label{fig:reponse_modale_prefiltre}
	\end{figure}
	
	
	\section{Gestion de l'erreur statique : Ajout d'un intégrateur}
	Même si cette réponse, obtenue dans le cas linéaire, se satisfait de l'usage d'une simple précommande, le sujet suggère l'ajout d'un intégrateur pour annuler l'erreur statique de vitesse~\cite[3.6.2]{Sujet} dans le cas non-linéaire. En effet, nous constatons que pour le modèle non-linéaire, la précommande seule amène le système en divergence. Cela s'explique par la grande variation de la densité de l'air qui rend la linéarisation peu valable lors d'une longue descente.
	Nous avons donc augmenté le système avec un état intégrateur $\xi$ défini par :
	\begin{equation}
		\dot{\xi} = V_{\text{EAS,cons}} - V_{\text{EAS,mes}}
	\end{equation}
	
	On choisit pour l'intégrateur un pôle à $-10.33$, ce qui assure une élimination rapide de l'erreur statique, et vérifie la contrainte de stabilité imposée ($\text{Re} < -1.3$).
	
	Le schéma de la commande modale avec intégrateur appliquée au système linéaire est illustré ci-dessous :
	
	\begin{figure}[H]
		\centering
		\makebox[\textwidth][c]{\includegraphics[width=1.2\textwidth]{schema_lineaire_modale.png}}
		\caption{Schéma Simulink de la boucle de vitesse avec intégrateur sur le système linéaire}
		\label{fig:schema_lineaire_modale}
	\end{figure}
	
	\section{Modèle non linéaire}
	
	Le schéma suivant illustre l'intégration de la commande modale sur le système non linéaire. Nous avons conservé la possibilité d'alterner entre la précommande $H$ et l'intégrateur, même si l'intégrateur se révèle être supérieur. Il est aussi important de noter que l'intégrateur agit sur l'erreur de la TAS, même si le système cherche à réguler la EAS. Nous faisons ce choix car la TAS est indépendante de l'altitude et donc plus facile à stabiliser.
	
	\begin{figure}[H]
		\centering
		\makebox[\textwidth][c]{\includegraphics[width=\textwidth]{schema_nonlineaire_modale.png}}
		\caption{Schéma de la commande modale appliquée au modèle non linéaire}
		\label{fig:schema_modale_nonlineaire}
	\end{figure}
	
	\section{Résultats de simulation}
	
	Le graphique suivant présente la réponse du modèle non linéaire à un échelon d'amplitude \SI{2}{kts} sur une durée de \SI{60}{s}. Le dépassement de l'ordre de \SI{0,2}{\%} et le temps de réponse de \SI{6}{s} sont adéquats. L'erreur statique relative est de l'ordre de \SI{0,1}{\%}, ce qui est suffisant pour un contrôle de la vitesse au noeud près. On peut donc conclure que le comportement de la loi de maintien de la vitesse sur le modèle non linéaire est satisfaisant.
	
	\begin{figure}[H]
		\centering
		\includegraphics[width=\textwidth]{resultat_vitesse.png}
		\caption{Réponse du système non linéaire à un échelon de consigne d'amplitude \SI{2}{kts}}
		\label{fig:resultat_vitesse}
	\end{figure}
	
	%----------------------------------------------------------
	\chapter{Lois de maintien du niveau et de la vitesse}
	%----------------------------------------------------------
	
	Pour la phase de stabilisation au niveau 30, l'avion doit maintenir simultanément son altitude et sa vitesse aérodynamique de \SI{230}{kts}. Il s'agit d'un problème multi-variable (MIMO) où la poussée $\delta_{thr}$ et la profondeur $\delta_{el}$ sont utilisées conjointement.
	
	\section{Linéarisation}
	
	Pour mettre en œuvre la commande, on linéarise le modèle avion au nouveau point de trim : \SI{230}{kts} et \SI{3000}{ft}. Cette linéarisation est réalisée par \texttt{utComputeTrimLevelFlight.m} et donne le point de fonctionnement suivant :
	\begin{equation}
		\text{trimValLF} = \left( \begin{array}{c}
			\delta_{\text{PHR}} \\
			\alpha_{\text{trim}} \\
			\delta_{\text{THR}}
		\end{array} \right) =
		\left( \begin{array}{c}
			\SI{-10.0}{\degree} \\
			\SI{3.6}{\degree} \\
			\SI{0.25}{\%}	
		\end{array} \right)
	\end{equation}
	
	\section{Synthèse par commande optimale (LQT)}
	Nous avons implémenté une structure \textit{Linear Quadratic Tracker} (LQT) pour minimiser l'erreur de suivi entre la sortie du système $y$ et un modèle de référence $y_r$. Le critère à minimiser est défini par :
	\begin{equation}
		J = \frac{1}{2} \int_{0}^{\infty} (e^T Q e + \tilde{u}^T R \tilde{u}) dt
	\end{equation}
	Où $Q$ et $R$ sont les matrices de pondération permettant de régler le compromis entre précision du suivi et énergie de commande.
	
	\section{Réglage des matrices de pondération}
	Pour respecter les spécifications (temps de réponse $< \SI{20}{s}$ pour $V_a$ et $< \SI{2}{s}$ pour la pente, sans dépassement), nous avons choisi :
	\begin{itemize}
		\item \textbf{Matrice $Q$} : Une pondération forte a été placée sur l'erreur de pente pour garantir la réactivité de la boucle d'altitude. On a choisi les composantes de $Q$ de l'ordre de 1000.
		\item \textbf{Matrice $R$} : Les coefficients ont été ajustés pour éviter des sollicitations brutales des moteurs (limitation de la fatigue) et des gouvernes. On a choisi les composantes de $R$ de l'ordre de 1.
	\end{itemize}
	
	\begin{figure}[H]
		\centering
		\makebox[\textwidth][c]{\includegraphics[width=\textwidth]{schema_lineaire_optimale.png}}
		\caption{Schéma de la commande optimale appliquée au modèle linéaire}
		\label{fig:schema_lineaire_optimale}
	\end{figure}
	
	\section{Analyse du couplage}
	Contrairement aux lois SISO précédentes, la commande optimale LQT permet un \textbf{découplage} partiel des axes. Une variation de consigne de vitesse est compensée par une action combinée sur la poussée (action directe) et sur la profondeur (pour contrer le moment cabreur induit par les moteurs). Le graphique suivant illustre comment répond le modèle linéaire grâce à ce découplage.
	
	\begin{figure}[H]
		\centering
		\makebox[\textwidth][c]{\includegraphics[width=\textwidth]{reponse_lineaire_optimale.png}}
		\caption{Réponse de la commande optimale appliquée au modèle linéaire à des échelons de pente de \SI{10}{\degree} et une réduction de vitesse de \SI{260}{kts} à \SI{230}{kts}}
		\label{fig:reponse_lineaire_optimale}
		
		% AJOUT DU NOTA BENE ICI
		\vspace{0.2cm} % Un petit espace vertical
		\begin{minipage}{0.9\textwidth} % Pour centrer et limiter la largeur du texte
			\footnotesize
			\textbf{N.B. :} Les courbes de vitesse et de pente sont superposées sur la même figure car leurs valeurs numériques sont du même ordre de grandeur. Il convient toutefois de distinguer les unités : la vitesse est en nœuds (kts) et la pente en degrés (\si{\degree}).
		\end{minipage}
	\end{figure}

	%----------------------------------------------------------
	\chapter{Simulation du scénario complet}
	%----------------------------------------------------------
	
	\section{Gestion de la transition}
	Le passage de la loi \textit{Open Descent} (vitesse tenue par l'assiette) aux lois de maintien (vitesse tenue par la poussée) constitue une phase critique du vol. On fait remarquer que la loi \textit{Level Flight} est cette fois utilisée sur une dynamique d'avion non-linéaire. La transition entre les lois de commande est faite via un bloc \texttt{Switch} qui bascule lorsque l'altitude de l'avion passe sous la barre de \SI{914}{m} (\SI{3000}{ft}). Ceci implique que l'avion se stabilise environ \SI{50}{ft} sous son altitude cible. On peut corriger cela naïvement en réglant le seuil du bloc \texttt{Switch} (un seuil à \SI{17}{m} fonctionne), ou en implémentant une loi qui gère plus finement la capture de l'altitude, par connaissance de la vitesse verticale et de la durée de la manœuvre.
	
	\section{Intégration des différentes lois de commandes}
	
	\begin{figure}[H]
		\centering
		\makebox[\textwidth][c]{\includegraphics[width=1.2\textwidth]{schema_nonlineaire_final.png}}
		\caption{Schéma de l'intégration de toutes les lois de commande}
		\label{fig:schema_final}
	\end{figure}
	
	
	\begin{figure}[H]
		\centering
		\begin{subfigure}[b]{\textwidth}
			\includegraphics[width=\textwidth]{schema_OpenDescent_final.png}
			\caption{Schéma de la commande Open Descent}
			\label{fig:schema_OpenDescent_final}
		\end{subfigure}
		\begin{subfigure}[b]{0.6\textwidth}
			\includegraphics[width=\textwidth]{schema_PID_final.png}
			\caption{Schéma du contrôleur PID}
			\label{fig:schema_PID_final}
		\end{subfigure}
		\begin{subfigure}[b]{\textwidth}
			\includegraphics[width=\textwidth]{schema_LF_final.png}
			\caption{Schéma de la commande Level Flight}
			\label{fig:schema_LF_final}
		\end{subfigure}
		\caption{Schémas des différentes commandes internes}
		\label{fig:schemas_commandes_final}
	\end{figure}
	
	\pagebreak
	\section{Analyse de la trajectoire et des vitesses}
	La simulation complète montre une descente fluide à \SI{260}{kts} EAS. Lors de l'approche du FL30 :
	\begin{enumerate}
		\item La loi de capture d'altitude réduit la pente.
		\item La manette des gaz sort de la position \textit{Idle} pour maintenir la vitesse.
		\item L'avion se stabilise aux nouvelles consignes sans dépassement excessif, validant la robustesse des correcteurs sur le modèle non-linéaire.
	\end{enumerate}
	
	\
	
	\begin{figure}[H]
		\centering
		\begin{subfigure}[b]{0.8\textwidth}
			\includegraphics[width=\textwidth]{reponse_ALT_400.png}
			\caption{Courbe de suivi de l'altitude en mètres}
			\label{fig:reponse_finale_ALT}
		\end{subfigure}
		\begin{subfigure}[b]{0.8\textwidth}
			\includegraphics[width=\textwidth]{reponse_EAS_294-314.png}
			\caption{Courbe de suivi de la vitesse à la capture du niveau FL30}
			\label{fig:reponse_finale_EAS_short}
		\end{subfigure}
		\caption{Courbes d'altitude et de vitesse pendant la descente au niveau FL30}
		\label{fig:reponse_descente_final}
	\end{figure}
	
	\section{Analyse des efforts de commande}
	
	Pour compléter l'analyse des performances, il est indispensable de vérifier l'évolution des trois entrées de commande lors du scénario complet. L'objectif est de s'assurer que les sollicitations restent dans le domaine admissible des actionneurs et ne présentent pas de comportements aberrants.
	
	\begin{figure}[H]
		\centering
		
		% --- BLOC 1 : PHR ---
		\begin{minipage}[c]{0.48\textwidth}
			\centering
			\includegraphics[width=\linewidth]{actionneurs_final_PHR.png}
		\end{minipage}
		\hfill
		\begin{minipage}[c]{0.48\textwidth}
			\textbf{Plan Horizontal Réglable ($\delta_{PHR}$)} \\
			L'évolution de la commande du PHR suit logiquement les changements de point de trim au cours de la descente. On constate que ses variations s'effectuent par paliers lors des transitions entre les différentes phases de vol, ce qui correspond au comportement attendu pour un organe d'équilibrage à dynamique lente.
		\end{minipage}
		
		\vspace{0.5cm} % Espacement vertical entre les blocs
		
		% --- BLOC 2 : GAZ ---
		\begin{minipage}[c]{0.48\textwidth}
			\centering
			\includegraphics[width=\linewidth]{actionneurs_final_THR.png}
		\end{minipage}
		\hfill
		\begin{minipage}[c]{0.48\textwidth}
			\textbf{Commande de poussée ($\delta_{thr}$)} \\
			On observe des oscillations d'environ \SI{10}{Hz} et de forte amplitude. Si l'inertie mécanique protège le moteur, ce comportement est inacceptable pour le FADEC qui risquerait de se mettre en sécurité. Ce phénomène suggère un réglage trop lâche de la matrice $R$ dans la commande optimale : le coût énergétique est sous-évalué. Un réglage plus fin lisserait cette commande.
		\end{minipage}
		
		\vspace{0.5cm} % Espacement vertical entre les blocs
		
		% --- BLOC 3 : ELEVATORS ---
		\begin{minipage}[c]{0.48\textwidth}
			\centering
			\includegraphics[width=\linewidth]{actionneurs_final_elevator.png}
		\end{minipage}
		\hfill
		\begin{minipage}[c]{0.48\textwidth}
			\textbf{Gouverne de profondeur ($\delta_{el}$)} \\
			La gouverne a un comportement cohérent mais atteint brutalement sa saturation lors de la ressource pour la capture de pente. Cette sollicitation extrême résulte de la spécification imposant un temps de réponse de \SI{2}{s}. Bien que le cahier des charges soit respecté, une telle dynamique pourrait s'avérer dangereuse structurellement ou inconfortable.
		\end{minipage}
		
		\caption{Analyse détaillée des trois actionneurs lors de la simulation complète}
		\label{fig:analyse_actionneurs_sidebyside}
	\end{figure}
	
	%----------------------------------------------------------
	\chapter{Conclusion}
	%----------------------------------------------------------
	
	Ce projet a permis de concevoir et valider une architecture de pilotage automatique complète pour la descente et la tenue de palier d'un Airbus A320. Ce scénario est le même que Airbus résout avec la loi de commande \texttt{Alt *}. La démarche progressive, allant de la linéarisation à la validation non-linéaire, a permis de mettre en évidence plusieurs points clés de l'automatique aéronautique :
	
	\begin{itemize}
		\item \textbf{Gestion des non-linéarités :} La linéarisation autour d'un point de trim est une méthode efficace pour la synthèse, mais elle montre ses limites lors des phases de grande amplitude (\textit{Open Descent}). L'ajout d'intégrateurs s'est révélé indispensable pour compenser les variations paramétriques (notamment la densité de l'air) et annuler l'erreur statique du modèle réel.
		\item \textbf{Apport de la commande multivariable :} La commande optimale (LQT) a offert une gestion supérieure des interactions entre la vitesse et la pente lors de la capture du niveau 30, assurant un découplage efficace difficile à obtenir avec des boucles SISO indépendantes.
	\end{itemize}
	
	\paragraph{Analyse critique}
	L'étude détaillée des signaux de commande lors de la simulation finale a cependant révélé les limites des réglages actuels vis-à-vis des contraintes physiques :
	\begin{itemize}
		\item \textbf{Compromis Performance/Sécurité :} Le respect strict du cahier des charges sur la capture de pente ($t_s < \SI{2}{s}$) contraint la gouverne de profondeur à la saturation. Bien que mathématiquement valide, cette dynamique est irréaliste d'un point de vue opérationnel (confort passager et contraintes structurelles).
		\item \textbf{Sollicitation moteur :} Des oscillations importantes ont été observées sur la commande de poussée. Elles indiquent un réglage trop faible de la pénalité sur l'énergie de commande (matrice $R$) qui, bien que filtrées par l'inertie mécanique, pourraient être rejetées par les sécurités du calculateur moteur (FADEC).
	\end{itemize}
	
	\paragraph{Perspectives}
	Pour fiabiliser cette loi de commande en vue d'une implémentation réelle, il conviendrait de relâcher la contrainte temporelle sur la capture de pente, d'augmenter le poids de la commande moteur dans le critère d'optimisation LQT, et d'intégrer explicitement les saturations de vitesse des gouvernes dans le modèle de synthèse.

\end{document}
